\section*{Chapitre \ref{chap:selection}}
\addcontentsline{toc}{section}{Chapitre \protect\ref{chap:selection}}

\begin{solution}{4.1}
\begin{enumerate}
\item
\begin{enumerate}
\item modèle D
\item modèle D
\item modèle G
\item modèle G
\item modèle C
\item modèle H
\end{enumerate}

\item Il y a un très gros problème de multicolinéarité pour les modèle F, G et H, car certains VIFs sont beaucoup plus grands que 10. Ce problème augmente inutilement la variance des paramètres estimés.

\item On évite les modèles F G et H pour ne pas avoir de problème de multicolinéarité. Le modèle D est préférable selon les critères PRESS et $R^2_p$. De plus, ses critères AIC et BIC sont les deuxièmes plus petits. Le $C_p$ est 8, donc 8-5=3. Ce n'est pas parfait, mais ce n'est pas si mal, etc.

\end{enumerate}
\end{solution}
\begin{solution}{4.2}
\begin{enumerate}
\item Puisque $n=p$, $\beta_0=0$ et que la matrice d'incidence est diagonale, on a $\hat{y}_i = \hat\beta_i$ pour $i=1,\ldots,n$. On minimise $S(\bobeta)=\sum_{i=1}^n (y_i-\beta_i)^2$ et on trouve pour $i\in \{1,\ldots,n\}$,
$$
\left.\frac{\partial}{\partial\beta_i}S(\bobeta)\right|_{\hat\beta_i}= -2 (y_i-\hat\beta_i) =0 \quad \Rightarrow \quad \hat\beta_i =  y_i.
$$

\item On minimise, pour une valeur $\lambda>0$,
$$
S^{\mathrm{ridge}}(\bobeta)=\sum_{i=1}^n (y_i-\beta_i)^2 +\lambda \sum_{i=1}^n \beta_i^2.
$$

\item On a
$$
\frac{\partial}{\partial\beta_i}S^{\mathrm{ridge}}(\bobeta)=-2(y_i-\beta_i) +2 \lambda  \beta_i.
$$
On pose égal à 0 et on trouve
$$
y_i-\hat \beta_i^{\mathrm{ridge}} = \lambda  \hat\beta_i^{\mathrm{ridge}} \quad \Rightarrow \quad \hat\beta_i^{\mathrm{ridge}} =  \frac{y_i}{1+\lambda}.
$$

\item On minimise, pour une valeur $\lambda>0$,
$$
S^{\mathrm{lasso}}(\bobeta)=\sum_{i=1}^n (y_i-\beta_i)^2 +\lambda \sum_{i=1}^n |\beta_i|.
$$

\item On a
$$
\frac{\partial}{\partial\beta_i}S^{\mathrm{lasso}}(\bobeta)=-2(y_i-\beta_i) + \lambda\, \mathrm{signe}(\beta_i) .
$$
On utilise les EMV trouvés en a) pour définir le signe. Supposons d'abord que $\hat\beta_i = y_i>0$. Alors, on a aussi $\hat\beta_i^{\mathrm{lasso}}>0$ (sinon, changer le signe donnera une valeur plus petite de l'équation à minimiser). On pose la dérivée égale à 0 et on trouve
$$
2(y_i-\hat \beta_i^{\mathrm{lasso}}) = \lambda  \quad \Rightarrow \quad \hat\beta_i^{\mathrm{ridge}} =  y_i- \lambda/2,
$$
ce qui tient seulement si $\hat\beta_i^{\mathrm{lasso}}>0$, alors on a $\hat\beta_i^{\mathrm{ridge}} =  \max(0,y_i- \lambda/2)$.
Supposons ensuite que $\hat \beta_i = y_i <0$. Alors, on a aussi $\hat\beta_i^{\mathrm{lasso}}<0$. On pose la dérivée égale à 0 et on trouve
$$
2(y_i-\hat \beta_i^{\mathrm{lasso}}) = -\lambda  \quad \Rightarrow \quad \hat\beta_i^{\mathrm{ridge}} =  y_i+ \lambda/2,
$$
sous la contrainte que ce soit négatif, donc dans ce cas, $\hat\beta_i^{\mathrm{ridge}} =  \min(0,y_i+\lambda/2)$. On combine les deux cas et on obtient l'équation donnée.

\item On peut voir que la façon de rapetisser les paramètres est bien différente pour les deux méthodes. Avec ridge, chaque coefficient des moindres carrés est réduit par la même proportion. Avec lasso, chaque coefficient des moindres carrés est réduit vers 0 d'un montant constant $\lambda/2$; ceux qui sont plus petits que $\lambda/2$ en valeur absolue sont mis exactement égaux à 0. C'est de cette façon que le lasso permet de faire la sélection des variables explicatives.

\end{enumerate}
\end{solution}
